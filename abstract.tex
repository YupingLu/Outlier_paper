\begin{abstract}

The Atmospheric Radiation Measurement (ARM) Data Center at ORNL collects data
from a number of permanent and mobile facilities around the globe. The
data is then ingested to create high level scientific products. High
frequency streaming measurements from sensors and radar instruments at
ARM sites requires high degree of accuracy to enable rigorous study of
atmospheric processes. Outliers in collected data are common 
due to instrument failure or extreme weather events. Thus, it is critical 
to identify and flag them. We employed multiple univariate, multivariate
and time series techniques for outlier detection methods and 
studied their effectiveness. First, we examined Pearson correlation coefficient 
which is used to measure the pairwise correlations between variables. 
Singular Spectrum Analysis (SSA) was applied to detect outliers by removing
the anticipated annual and seasonal cycles from the signal to accentuate 
anomalies. K-means was applied for multivariate examination of data
from collection of sensor to identify any deviation from expected and
known patterns and identify abnormal observation. The Pearson correlation 
coefficient, SSA and K-means methods were later combined together in a 
framework to detect outliers through a range of checks. We applied the
developed method to data from meteorological sensors at ARM Southern
Great Plains site and validated against existing database of known data
quality issues. %Compared to the current 181 anomaly entries stored in the Data Quality Report database, the framework detected 1052 outliers which is 5.8 times more.

\end{abstract}
