%%%%%%%%%%%%%%%%%%%%%%%%%%%%%%%%%%%%%%%%%%%%%%%%%%%%%%%%%%%%%%%%%%%%%%%%%%%%%%%%
%2345678901234567890123456789012345678901234567890123456789012345678901234567890
%        1         2         3         4         5         6         7         8

\documentclass[letterpaper, 10 pt, conference]{ieeeconf}  % Comment this line out
                                                          % if you need a4paper
%\documentclass[a4paper, 10pt, conference]{ieeeconf}      % Use this line for a4
                                                          % paper
\IEEEoverridecommandlockouts                              % This command is only
                                                          % needed if you want to
                                                          % use the \thanks command
%\overrideIEEEmargins
% See the \addtolength command later in the file to balance the column lengths
% on the last page of the document

% The following packages can be found on http:\\www.ctan.org
%\usepackage{graphics} % for pdf, bitmapped graphics files
%\usepackage{epsfig} % for postscript graphics files
%\usepackage{mathptmx} % assumes new font selection scheme installed
%\usepackage{times} % assumes new font selection scheme installed
%\usepackage{amsmath} % assumes amsmath package installed
%\usepackage{amssymb}  % assumes amsmath package installed

\title{\LARGE \bf
Outlier Detection for ARM Data
}

\author{Yuping Lu$^{1}$, Jitendra Kumar$^{2}$ and Michael A. Langston$^{1}$% <-this % stops a space
\thanks{$^{1}$University of Tennessee, Knoxville, TN, USA}%
\thanks{$^{2}$Oak Ridge National Laboratory, Oak Ridge, TN, USA}%
}

\begin{document}
\maketitle
\thispagestyle{empty}
\pagestyle{empty}

%%%%%%%%%%%%%%%%%%%%%%%%%%%%%%%%%%%%%%%%%%%%%%%%%%%%%%%%%%%%%%%%%%%%%%%%%%%%%%%%
\begin{abstract}

Add outlier detection abstract here.

\end{abstract}


%%%%%%%%%%%%%%%%%%%%%%%%%%%%%%%%%%%%%%%%%%%%%%%%%%%%%%%%%%%%%%%%%%%%%%%%%%%%%%%%
\section{Introduction}
We will use this section to introduce the background of outlier detection for 
time series data. \cite{gupta2014outlier} Also mention the problem in ARM data 
here?


\section{Datasets}
Atmospheric Radiation Measurement (ARM) user facility data are collected through 
routine operations and scientific field experiments. ARM data include routine 
data products, value-added products (VAPs), field campaign data, complementary 
external data products from collaborating programs, and data contributed by ARM 
principal investigators for use by the scientific community. Data quality 
reports, graphical displays of data availability/quality, and data plots are 
also available from the ARM Data Center.

We tested 24 datasets and 5 variables which are \textit{temp\_mean}, 
\textit{vapor\_pressure\_mean}, \textit{atmos\_pressure}, 
\textit{rh\_mean} and \textit{wspd\_arith\_mean}.

\begin{table}[h]
\caption{SGPMET datasets tested}
\label{tab:template}
\centering
\begin{tabular}{|l|c|c|c|c|c|c|c|c|}
\hline
Instrument & E1 & E3 & E4 & E5 & E6 & E7\\
Begin Year & 1996 & 1997 & 1996 & 1997 & 1997 & 1996\\
End Year & 2008 & 2008 & 2010 & 2008 & 2010 & 2011\\
\hline
Instrument & E8 & E9 & E11 & E13 & E15 & E20\\
Begin Year & 1994 & 1994 & 1996 & 1994 & 1994 & 1994\\
End Year & 2008 & 2017 & 2017 & 2017 & 2017 & 2010\\
\hline
Instrument & E21 & E24 & E25 & E27 & E31 & E32\\
Begin Year & 2000 & 1996 & 1997 & 2004 & 2012 & 2012\\
End Year & 2017 & 2008 & 2001 & 2009 & 2017 & 2017\\
\hline
Instrument & E33 & E34 & E35 & E36 & E37 & E38\\
Begin Year & 2012 & 2012 & 2012 & 2012 & 2012 & 2012\\
End Year & 2017 & 2017 & 2017 & 2017 & 2017 & 2017\\
\hline
\end{tabular}
\end{table}

%var_names = ['temp_mean', 'vapor_pressure_mean', 'atmos_pressure', 'rh_mean', 'wspd_arith_mean']

\section{Methodology}
Mention methods we used in this paper and how do we preprocess the data.

\subsection{Pearson Correlation Coefficient} 
PCC goes here \cite{pearson1895note}.

\subsection{Singular Spectrum Analysis}
SSA goes here \cite{golyandina2013singular, alexandrov2008method}.

\subsection{K-means}
k-means goes here \cite{hartigan1979algorithm}.


\section{Results and Discussion}
Results and pics go here. Comparison metric: DQR database. Add precision and 
recall result here \cite{perry1955machine}.


\section{Conclusions}
We presented a combined model to detect outliers for ARM data. Future work: 
ML and tried methods working on multiple instruments multiple sites.

%\addtolength{\textheight}{-12cm}  % This command serves to balance the column lengths
                                  % on the last page of the document manually. It shortens
                                  % the textheight of the last page by a suitable amount.
                                  % This command does not take effect until the next page
                                  % so it should come on the page before the last. Make
                                  % sure that you do not shorten the textheight too much.

%%%%%%%%%%%%%%%%%%%%%%%%%%%%%%%%%%%%%%%%%%%%%%%%%%%%%%%%%%%%%%%%%%%%%%%%%%%%%%%%
\section*{Acknowledgment}
This research was supported by the Atmospheric Radiation Measurement (ARM) user 
facility, a U.S. Department of Energy (DOE) Office of Science user facility 
managed by the Office of Biological and Environmental Research.


\bibliography{main} 
\bibliographystyle{IEEEtran}


\end{document}