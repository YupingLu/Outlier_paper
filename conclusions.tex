\section{Conclusions}
In this paper we tested pairwise Pearson correlation,
univariate SSA and multivariate $k$-means based method for detection of
outliers in the data at ARM meteorological observations at SGP site. 
Combining the approaches within a framework for
streaming data within ARM provides a platform to detect outliers
from a wide range of sensor failure scenarios to extreme events.
While each of the methods developed and applied in this study has its
strengths and limitations, our evaluation against existing database of
data quality issue suggests that the framework is able to identify known
outliers well. Although our current study focused on
meteorological observations, it provides a framework for an efficient
outlier detection of streaming datasets within ARM that can be extended to
other classes of time series datasets not only tested MET data from SGP. 
In the future, we plan to analyze multiple
classes of instruments like meteorological, radiometric, radar etc.
simultaneously for improved detection of outliers. We also plan to
develop multivariate SSA \cite{rodrigues2018benefits} and machine learning 
techniques to address this high dimensional problem in an operational 
data center environment.

The three algorithms and visualizations presented in this paper were 
implemented in Python. All codes and results are available on GitHub 
(https://github.com/YupingLu/arm-pearson and https://github.com/YupingLu/arm-ssa). 
