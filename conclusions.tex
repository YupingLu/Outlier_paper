\section{Conclusions}
In this paper we tested pairwise Pearson correlation,
univariate SSA and multivariate $k$-means based method for detection of
outliers in the data at ARM meteorological observations at SGP site. 
Combing the approaches within a framework that can be applied to
streaming data set within ARM provied an platform to detect outliers
from a wide range of sensor failure scenarios to extreme events.
While each of the method developed and applied in this study had their
strengths and limitations, our evaluation against existing database of
data quality issue suggest that the framework was able to identify known
outliers well and many more. While our current study focused on
meteorological observations, it provides a framework for an efficient
outlier detection of streaming datasets within ARM that be extended to
other classes of time series dataset.
only tested MET data from SGP. In future we plan to analyze multiple
classes of intruments like meterological, radiometric, radar etc.
simultaneously for improved detection of outliers. We also plan to
develop graph theory and machine learning techniques to address this
high dimensional problem in an operational data center environment. 
instrument independently. We will apply this framework on other types
of data from other facilities in the future. Meanwhile, other methods
will be examined to test data from multiple instruments together such
as graph theory methods \cite{phillips2015graph} and machine learning
methods. 


% mention the usage of python and implementation. % add three sigma rule here.
The three algorithms and visualizations presented in this paper are 
implemented in Python. All codes and results are available on GitHub 
(https://github.com/YupingLu/arm-pearson and https://github.com/YupingLu/arm-ssa). 
