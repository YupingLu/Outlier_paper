\section{Datasets}
ARM data are stored and distributed in the Network Common Data Form (NetCDF) format which
is self-describing and machine-independent \cite{rew1990netcdf, NetCDF}
and has good performance and data compression. It is commonly used to
handle scientific data, especially in climate and Earth sciences,
meteorology, oceanography, and remote sensing etc. 
All ARM data are publicly available and can be downloaded from ARM Data Center
(https://www.arm.gov/data) where a large range of datasets ranging  
from meteorology, to atmospheric profiles, to weather radars to
satellite observations are available. Datasets are collected at a number
of different locations using large number of diverse
instruments are available within ARM. 

\begin{table}[ht]
\caption{SGPMET datasets used in this study}
\label{tab:datasets}
\centering
\begin{tabular}{|l|c|c|c|c|c|c|c|c|}
\hline
Facility & E1 & E3 & E4 & E5 & E6 & E7\\
Begin Year & 1996 & 1997 & 1996 & 1997 & 1997 & 1996\\
End Year & 2008 & 2008 & 2010 & 2008 & 2010 & 2011\\
\hline
Facility & E8 & E9 & E11 & E13 & E15 & E20\\
Begin Year & 1994 & 1994 & 1996 & 1994 & 1994 & 1994\\
End Year & 2008 & 2017 & 2017 & 2017 & 2017 & 2010\\
\hline
Facility & E21 & E24 & E25 & E27 & E31 & E32\\
Begin Year & 2000 & 1996 & 1997 & 2004 & 2012 & 2012\\
End Year & 2017 & 2008 & 2001 & 2009 & 2017 & 2017\\
\hline
Facility & E33 & E34 & E35 & E36 & E37 & E38\\
Begin Year & 2012 & 2012 & 2012 & 2012 & 2012 & 2012\\
End Year & 2017 & 2017 & 2017 & 2017 & 2017 & 2017\\
\hline
\end{tabular}
\end{table}

In this study, we used the data from Surface Meteorology Systems (MET)
collected at the ARM Southern Great Plains (SGP) site in
Oklahoma, United States. SGP is ARM's largest facility that
comprises of a network of core and extended facilities. In our study we
used MET data from 24 extended facilities where surface meteorological
observations have been collected continuously and independently. 
While MET instruments collect a large array of direct and indirect
measurements, we focused our analysis on five core meteorological variables:
air temperature (\textit{temp\_mean}), vapor pressure
(\textit{vapor\_pressure\_mean}),
atmospheric pressure (\textit{atmos\_pressure}), relative humidity
(\textit{rh\_mean}) and wind speed
(\textit{wspd\_arith\_mean}). These five core meteorological variables are
inputs for a large number of derived datasets produced by the ARM and
are often essential set of data for most atmospheric analysis, hence
focus of our study. Table~\ref{tab:datasets} provides details of sites
and available time series for the datasets used.
